\begin{resumo}[Abstract]
 \begin{otherlanguage*}{english}
The number of participation initiatives aimed at bridging the gap between governments and citizens has been increasing .
The concepts of digital governance, electronic government and electronic participation has entered the agenda of large organizations that seek the development of society.
In this scenario, the use of electronic participation tools has become a great way to monitor governments, share data and resolve the demands of society.
In line with this, this paper has developed a web application, called e-TAPE, to support a collaborative edition of a taxonomy on electronic participation tools.
The goal of e-TAPE is to facilitate user interaction intuitively and user-friendly and functions as a way of researchers and users collaborate for the evolution of the taxonomy.
The usability of e-TAPE is evaluated, following methodologies already validated in the academia and the results that were found has shown that e-TAPE has a good usability index.

\textbf{Key-words}: Taxonomy, eletronic participation, eletronic participation tools, TAPE, e-TAPE, e-gov and digital government.
 \end{otherlanguage*}
\end{resumo}
\newpage