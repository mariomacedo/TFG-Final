\newpage
% resumo em português
\begin{resumo}
Nota-se a aumento do número de iniciativas de participação com o objetivo de estreitar a distância entre governos e cidadãos. Os conceitos de governança digital, governo eletrônico
e participação eletrônica entraram na agenda de grandes organizações que buscam o desenvolvimento da sociedade. Nesse contexto, a utilização de ferramentas de participação eletrônica
têm se tornado um grande meio para a fiscalizar governos, compartilhar dados e resolver as demandas da sociedade.
Alinhado a isso, o objetivo desse trabalho é desenvolver uma aplicação \textit{web}, denominada e-TAPE, para apoiar a edição colaborativa de uma taxonomia sobre 
ferramentas de participação eletrônica. A proposta é que a e-TAPE facilite a interação dos usuários de maneira intuitiva e amigável e funcione como um canal para que, 
tanto pesquisadores quanto potenciais usuários de ferramentas de participação eletrônica, colaborem para evolução da taxonomia. 
Na e-TAPE, os usuários têm acesso à taxonomia e às ferramentas de participação eletrônica já classificadas.
 A usabilidade da e-TAPE é avaliada, seguindo metodologias já validadas no meio acadêmico e os resultados indicam que, a e-TAPE possui um bom indicador de usabilidade perante aos 
 usuários que realizaram o teste. 
\end{resumo}
\newpage