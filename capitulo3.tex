\chapter[Ferramenta]{Ferramenta Taxonopart}
\label{cap:cap3}

\subsection{Funcionamento}
\label{sec:funcionamento}
A ferramenta Taxonopart foi desenvolvida como uma aplicação web, e encontra-se disponível em http://taxonopart.hopto.org.
Ao acessar o endereço da aplicação, o usuário encontra uma breve explicação sobre o contexto do projeto em que ela foi desenvolvida e uma descrição da taxonomia considerando os grupos, classes e subclasses propostas.

\par
Ao acessar a página Taxonomia, o usuário começa sua interação com a aplicação. Foram disponibilizados dois modos de visualização da taxonomia: radia ou horizontal ilustrados, respectivamente,nas Figuras X e Y.

%FIGURAS PARA TUDO?
Para classificar uma nova ferramenta, o usuário deve clicar no botão com o  sinal de mais, no canto inferior da tela, tendo feito isso,
o formulário para classificação da ferramenta é exibido, conforme ilustrado na Figura Z. O formulário foi dividido de acordo com os grupos definidos na taxonomia (sustentação, domínio, tecnologias e funcionalidades), sendo o único campo de preenchimento
obrigatório, o nome da ferramenta. Para finalizar a classificação da nova ferramenta, o usuário deverá clicar no botão salvar, na última seção do formulário. 

\par
Por se tratar de uma ferramenta colaborativa, o usuário é capaz de visualizar e editar qualquer ferramenta já classificada por outros usuários. 
Para fazer isso, o usuário pode usar o campo buscar ferramenta, localizado no canto superior esquerdo da tela, e então ao visualizar a ferramenta o botão editar estará 
no cando superior direito da tabela com os dados de classificação da ferramenta em questão.

\par
Ao clicar em algum nó da taxonomia, em qualquer modo de visualização, é exibido um quadro com o nome da classe, descrição e todas as ferramentas
com valor de classificados pela classe, ou subclasse em questão.

\subsection{Tecnologias Utilizadas}
\label{subsec:tecnologias}

A aplicação foi construída utilizando a plataforma de desenvolvimento Node.js. A escolha dessa plataforma justifica-se pelo conhecimento técnico da equipe
responsável pelo desenvolvimento da aplicação e pelo conceito fundamental da plataforma. Node.js é uma plataforma construída sobre o motor JavaScript do 
Google Chrome para facilmente construir aplicações de rede rápidas e escaláveis. Node.js usa um modelo de I/O direcionada a evento não bloqueante que o torna leve e eficiente,
ideal para aplicações em tempo real com grande volume de troca de dados através de dispositivos distribuídos (nodejs.org, 2018).

\par
Após análises, foi identificada a necessidade da utilização de um banco de dados orientado a documentos para a realizar a persistência dos dados da aplicação. 
Sendo assim, a utilização do banco de dados MongoDB apresentou-se como uma escolha confiável e eficaz. 
Por se tratar de uma ferramenta \textit{open source}, e por se integrar com a plataforma Node.js, MongoDB atendeu aos requisitos necessários para o desenvolvimento da aplicação.

\par
Para os componentes visuais da aplicação, o \textit{front-end}, a utilização do \textit{framework} responsivo Materialize (https://materializecss.com), 
mostrou-se uma opção elegante e prática para o desenvolvimento da aplicação. Combinando o Materialize com o \textit{framework} de templates EJS (https://ejs.co), 
foi possível desenvolver a aplicação de maneira escalável e leve.

\par
Para a implementação de ambas visualizações da taxonomia, radial e horizontal, foi utilizada a D3.js, uma biblioteca JavaScript para a manipulação de documentos 
baseada em dados (https://d3js.org).
