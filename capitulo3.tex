\chapter[Ferramenta]{Ferramenta e-TAPE}
\label{cap:cap3}
Este trabalho teve como proposta o desenvolvimento de uma aplicação web, chamada e-TAPE, cujo o objetivo consiste em
disponibilizar um ambiente para edição colaborativa e evolução da taxonomia TAPE.

\par
Para desenvolver a aplicação foi adotado o modelo de ciclo de vida incremental, descrito na seção seguinte.

\section {Metodologia de Desenvolvimento da e-TAPE}
\label{sec:desenvolvimento}
\par
Segundo \citeonline{sommerville2016software}, o modelo incremental é baseado na ideia de se desenvolver uma aplicação inicial,
realizar a coleta de \textit{feedback} de clientes, usuários e outras pessoas envolvidas, evoluir a aplicação em versões, até que o desenvolvimento dos requisitos 
levantados seja realizado.

\par
Primeiramente, como primeira tarefa da fase de projetos, foram definidos os atores do sistema. Para a versão inicial, foram definidos duas classes de atores, são elas: 
atores administradores e atores visitantes. Os atores administradores não possuem limitações quanto ao acesso aos dados, podendo visualizar, atualizar, remover ou criar ferramentas,
taxonomias e usuários. Já os usuários visitantes, podem somente, interagir com as ferramentas classificadas ou classificar novas ferramentas. 

\par
Cada versão entregue da aplicação deve incorporar novas funcionalidades para melhor atender os requisitos levantados. Geralmente, ainda segundo o autor, as primeiras funcionalidades
implementadas são escolhidas pelo grau de importância para o produto final. Esse modelo permite a avaliação da aplicação em versões iniciais de desenvolvimento, para assim saber 
se os requisitos estão sendo entregues. A figura \ref{fig:modelo-incremental} demonstra a abstração do modelo incremental.


\begin{figure}[!ht]
    \centering
    \includegraphics[scale=0.20]{./figuras/modelo_incremental.png}
    \caption{Modelo de desenvolvimento Incremental \citeonline{sommerville2016software}}
    \label{fig:modelo-incremental}
\end{figure}

\par
Para o desenvolvimento da aplicação, foram realizados o levantamento dos requisitos de cliente e a modelagem conceitual dos dados.
Depois de detalhados os requisito, foi elaborado um protótipo e definição dos atores da aplicação. O lista de requisitos para a aplicação está no Apêndice \ref{apendice:b} deste trabalho. 

\par
Finalizada a etapa de projeto, iniciou-se a etapa de implementação. 

A aplicação foi construída utilizando a plataforma de desenvolvimento Node.js. A escolha dessa plataforma justifica-se pelo conhecimento técnico da equipe
responsável pelo desenvolvimento da aplicação e pelo conceito fundamental da plataforma. Node.js é uma plataforma construída sobre o motor JavaScript do 
Google Chrome para desenvolver aplicações de rede rápidas e escaláveis. Node.js usa um modelo de I/O direcionada a evento não bloqueante que o torna
ideal para aplicações em tempo real com grande volume de troca de dados através de dispositivos distribuídos. 
\textit{Node.js} Disponível em: <https://nodejs.org>. Acesso em 01 out. 2018.

\par
Após análise do modelo conceitual dos dados e das relações entre as entidades, foi identificada a necessidade da utilização de um banco de dados orientado a documentos para a 
realizar a persistência dos dados da aplicação. 
O \acrfull{sgbd} MongoDB foi escolhido por se tratar de uma ferramenta \textit{open source} e por se integrar com a plataforma Node.js 
\textit{MongoDB} Disponível em: <https://mongodb.com>. Acesso em 01 out. 2018.

\par
O \textit{front-end} da aplicação, foi implementado com a utilização do \textit{framework} responsivo Materialize, pois esse, 
mostrou-se uma opção adequada para o desenvolvimento da aplicação. Combinando o Materialize com o \textit{framework} de templates EJS, 
foi possível desenvolver a aplicação de maneira escalável e rápida. \textit{Materialize} Disponível em: <https://materializecss.com>. Acesso em 01 out. 2018.
\textit{EJS} Disponível em: <https://ejs.co>. Acesso em 01 out. 2018.

\par
Para a criação das visualizações da taxonomia, foram definidos dois modos de visualização, o modo radial, chamado na aplicação de e-TAPE 360º, e o modo horizontal, 
chamado de e-TAPE Árvore. Para a implementação de ambas visualizações, foi utilizada a biblioteca D3.js, uma biblioteca JavaScript para a manipulação de documentos 
baseada em dados, por se tratar de uma tecnologia que possibilita a visualização interativa de dados.
\textit{D3js} Disponível em: <https://d3js.org>. Acesso em 01 out. 2018.

\par
O controle de versão da aplicação foi realizado através do sistema de controle de versão Git na plataforma GitLab, \textit{GitLab}. 
Disponível em <https://gitlab.com>. Acesso em 01 out. 2018.

\section{Funcionalidades disponíveis}
\label{sec:funcionamento}
A ferramenta e-TAPE foi desenvolvida como uma aplicação web está disponível em http://taxonopart.hopto.org:3001.

\par
Ao acessar o endereço da aplicação, o usuário visitante encontra uma breve explicação sobre o contexto do projeto e uma descrição da taxonomia considerando os grupos,
classes e subclasses propostas. É possível obter detalhes, sobre as classes e subclasses, clicando em seus respectivos nomes. 

\par
Ao acessar a página Taxonomia, o usuário visitante começa sua interação ativa com a aplicação.
Foram disponibilizados dois modos de visualização da taxonomia: o e-TAPE 360º e o e-TAPE Árvore, ilustrados, respectivamente,
nas Figuras \ref{fig:e-tape360} e \ref{fig:e-tapeArvore}. Podendo-se alterar entre os modos de visualização clicando no botão logo acima da visualização.
O botão pode ser visto na Figura \ref{fig:pag-taxonomia}.

\vspace{1cm}

\begin{figure}[!ht]
    \centering
    \includegraphics[scale=0.20]{./figuras/taxonomia-cropped.png}
    \caption{e-TAPE 360º}
    \label{fig:e-tape360}
\end{figure}

\vspace{1cm}

\begin{figure}[!ht]
    \centering
    \includegraphics[scale=0.20]{./figuras/taxonopart-horizontal.png}
    \caption{e-TAPE Árvore}
    \label{fig:e-tapeArvore}
\end{figure}
\newpage

\par
Caso queira visualizar uma ferramenta já classificada, o usuário deve utilizar o campo de pesquisa, no canto superior esquerdo da tela Taxonomia.
Conforme o usuário for digitando o nome da ferramenta, o campo sugere ferramentas pelo modo de preenchimento automático. Se o usuário escreveu o nome da ferramenta corretamente 
e a ferramenta já estiver classificada, o usuário pode visualizar uma tabela com a classificação da 
ferramenta, vide Figura \ref{fig:show-ferramenta}.

\vspace{0.5cm}

\begin{figure}[!ht]
    \centering
    \includegraphics[scale=0.20]{./figuras/show-ferramenta.png}
    \caption{Visualizar ferramenta classificada}
    \label{fig:show-ferramenta}
\end{figure}


\par
Para classificar uma nova ferramenta, o usuário deve clicar no botão com o  sinal de "+", no canto inferior da tela Taxonomia, ilustrada na Figura \ref{fig:pag-taxonomia}. Tendo feito isso, o formulário para classificação da ferramenta é exibido, conforme ilustrado na Figura \ref{fig:new-ferramenta}. 

\begin{figure}[!ht]
    \centering
    \includegraphics[scale=0.10]{./figuras/pagina-taxonomia.png}
    \caption{Página Taxonomia em e-TAPE }
    \label{fig:pag-taxonomia}
\end{figure}

\par
O formulário de classificação de novas ferramentas foi dividido de acordo com os grupos definidos na taxonomia 
(sustentação, domínio, tecnologias e funcionalidades), sendo o único campo de preenchimento obrigatório, o nome da ferramenta.
Para finalizar a classificação da nova ferramenta o usuário pode a qualquer momento clicar no botão salvar. 

\begin{figure}[!ht]
    \centering
    \includegraphics[scale=0.20]{./figuras/new-ferramenta.png}
    \caption{Formulário de classificação ferramentas}
    \label{fig:new-ferramenta}
\end{figure}

\par
Por se tratar de uma ferramenta colaborativa, o usuário visitante é capaz de visualizar e editar qualquer ferramenta já classificada por outros usuários. 
Para fazer isso, após visualizar a ferramenta, basta clicar no botão editar, que é representado pelo ícone de um lápis. 

\newpage
\par
Caso o usuário visitante queria saber a definição de uma classe, duas opções estão disponíveis, parar o mouse em cima de algum nó da taxonomia, em qualquer modo de visualização, 
ou clicar no nó desejado. Para a primeira opção é exibido um \textit{tooltip} com a definição do nó, para a segunda opção um quadro com o nome da classe, sua descrição e 
todas as ferramentas com valor de classificação atrelados à classe, ou subclasse é exibido. A figura \ref{fig:tabela-ferramentas} demonstra essa segunda ação. 

\begin{figure}[!ht]
    \centering
    \includegraphics[scale=0.20]{./figuras/abordagem.png}
    \caption{Tabela com ferramentas classificadas segundo Abordagem}
    \label{fig:tabela-ferramentas}
\end{figure}