\chapter[Introdução]{Introdução}
\label{cap:cap1}

% Falar sobre governance -> participação cidadã -> participação eletronica

% LESSONS LEARNED FROM THE SUCCESS OF COMPUTER PLATFORMS
% There is a new compact on the horizon: information produced by and on behalf of
% citizens is the lifeblood of the economy and the nation; government has a respon-
% sibility to treat that information as a national asset. Citizens are connected like
% never before and have the skill sets and passion to solve problems affecting them
% locally as well as nationally. Government information and services can be provided to citizens where and when they need them. Citizens are empowered to spark
% the innovation that will result in an improved approach to governance. In this
% model, government is a convener and an enabler rather than the first mover of
% civic action.

% As Governments are committing more effort to understand an increasingly
% interdependent and complex world [3], [25], [27], [32], citizens demand more
% openness, transparency and commitment to results [8] - within or after the financial
% crisis. Moreover, citizens are becoming increasingly vocal in monitoring and
% influencing policy decisions, through the new media [31]

% Falar sobre governance -> participação cidadã -> participação eletronica
\par
Atualmente, uma tendência que vem sendo observada é a tentativa de aproximação entre os cidadãos e seus governantes, um exemplo disso são as iniciativas de governos abertos.
A agenda de 2030 da \acrfull{onu} para a transformação do mundo através do desenvolvimento sustentável deixa claro, no parágrafo 48, que
indicadores estão sendo desenvolvidos para ajudar nesse trabalho de aproximar governo e cidadão. Dados desagregados de qualidade, acessíveis,
atualizados e confiáveis serão necessários para colaborar com a análise do progresso das iniciativas sobre participação. 
Esses dados são de suma importância para a tomada de decisões \cite{assembly2015transforming}.

\par
A \acrshort{onu} define o conceito de participação eletrônica como "o processo de engajar cidadãos através de \acrfull{tic} em política, tomadas de decisão e
serviços públicos, fazendo com que este processo seja inclusivo, participativo e deliberado". Em \citeonline{braga2016participaccao}, os autores mostram que o uso de ferramentas de participação eletrônica tornou-se fundamental para o cumprimento das metas traçadas em 2015. 

\par
A \acrfull{ocde}, juntamente com a \acrshort{onu} e a \acrfull{ue} criaram \textit{frameworks} para a avaliação do \textit{status} do governo eletrônico através de indicadores \cite{onu2018}. A metodologia utilizada permite medir a eficácia do governo eletrônico na prestação de serviços públicos e identificar padrões de desenvolvimento e desempenho.
Atualmente, governantes podem encontrar diretrizes para seguir na hora de desenvolver uma ferramenta de participação eletrônica. 
Por exemplo,  no trabalho de \citeonline{scherer2010hands}, os autores apresentam uma metodologia para a criação de duas ferramentas de participação eletrônica na Europa, a VoicE e a VoiceS.

%==================================================================================================================================================================================================
%Justificativa (uma classificação pode apoiar o entendimento nessa área)
Outra vertente de estudo nessa área tem sido o esforço de classificação de ferramentas de participação eletrônica. Em  \citeonline{kinyik2015guideline}, o autor apresenta um modelo de classificação elaborado no contexto do projeto \acrfull{e-uropa} cujo objetivo é aumentar e disseminar o conhecimento dos cidadãos europeus sobre ferramentas de participação eletrônica, alegando que assim, as demandas da sociedade seriam melhores e mais rapidamente atendidas.
\par
Contudo, apesar dos esforços de classificação, o conhecimento a repeito dessa tipo de ferramenta ainda é muito restrito. Dessa forma, é necessário a elaboração de estratégias que possam popularizar ou ampliar o alcance dessas iniciativas. 
Alinhado a essa necessidade, o objetivo desse trabalho foi desenvolver uma aplicação \textit{web} para apoiar a edição colaborativa de uma taxonomia sobre ferramentas
de participação eletrônica. A aplicação permitirá a interação dos usuários de maneira intuitiva e amigável, colaborando para a edição das informações apresentadas.
Essa aplicação apresentará tanto a taxonomia, quanto as ferramentas de participação eletrônica classificadas pelos usuários.

Espera-se construir uma ferramenta que contribua tanto para a pesquisa quanto para a prática do assunto abordado. No que diz respeito à pesquisa, a ferramenta pode ajudar os pesquisadores a entender mais sobre domínio. 
Espera-se que com a utilização de um modelo sistemático representado pela taxonomia, seja possível identificar pontos fortes e fracos criando novas possibilidades de investigação.
\par
Por outro lado, tanto para o gestor público, quanto para o cidadão comum, o uso da aplicação pode facilitar a escolha de uma ou mais ferramentas de participação
eletrônica que seja mais adequada ao seu contexto. Essa escolha poderá levar em conta critérios isolados ou um aglomerado de critérios, dependendo da demanda do usuário.
