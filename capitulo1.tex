\chapter[Introdução]{Introdução}
\label{cap:cap1}
Uma tendência que vem sendo observada é a aproximação entre os cidadãos e seus governantes, um exemplo disso são as iniciativas de governos abertos.
A agenda de 2030 da \acrfull{onu}, \citeonline{assembly2015transforming}, para a transformação do mundo através do desenvolvimento sustentável deixa claro, no parágrafo 48, que
indicadores estão sendo desenvolvidos para ajudar a estreitar a distância entre governo e cidadão. Dados acessíveis, atualizados e confiáveis serão necessários para colaborar 
com a análise do progresso das iniciativas sobre participação. 
Esses dados são de suma importância para a tomada de decisão \cite{assembly2015transforming}.

\par
A \acrshort{onu} define o conceito de participação eletrônica como "o processo de engajar cidadãos através de \acrfull{tic} em política, tomadas de decisão e
serviços públicos, fazendo com que este processo seja inclusivo, participativo e deliberado". Em \citeonline{braga2016participaccao}, os autores mostram que o uso de 
ferramentas de participação eletrônica tornou-se fundamental para o cumprimento das metas traçadas pela agenda da \acrshort{onu}, em 2015. 

\par
Isso dá-se pelo fato de que a utilização de \acrshort{tic} pelo setor público tem transformado a governança global, demandando que os agentes públicos forneçam melhores serviços 
de maneira eficiente e barata \cite{afdb2014uneca}. 

\par
Esse tipo de demanda provoca a adoção, pelo poder público, de infraestruturas tecnológicas para a melhora e eficácia ao ofertar serviços à população. Esse conceito é interpretado
por \citeonline{reddick2012public} como governança digital. Ou seja, governança digital é a melhora da capacidade do Estado, fazendo o uso de tecnologias e da internet 
para a implementação de suas políticas.

\par
A \acrfull{ocde}, juntamente com a \acrshort{onu} e a \acrfull{ue}, criaram \textit{frameworks} para a avaliação do \textit{status} do governo eletrônico através de indicadores \cite{onu2018}. A metodologia utilizada permite medir a eficácia do governo eletrônico na prestação de serviços públicos e identificar padrões de desenvolvimento e desempenho.
Atualmente, governantes podem encontrar diretrizes para seguir ao desenvolver uma ferramenta de participação eletrônica. 
Por exemplo,  no trabalho de \citeonline{scherer2010hands}, os autores apresentam uma metodologia para a criação de duas ferramentas de participação eletrônica na Europa, a VoicE e a VoiceS.

\par
Outra vertente de estudo nessa área tem sido o esforço de classificação de ferramentas de participação eletrônica. Em  \citeonline{kinyik2015guideline}, o autor apresenta um modelo 
de classificação elaborado no contexto do projeto \acrfull{e-uropa} cujo objetivo é aumentar e disseminar o conhecimento dos cidadãos europeus sobre ferramentas de participação 
eletrônica, alegando que assim, as demandas da sociedade seriam melhores e mais rapidamente atendidas. Contudo, apesar dos esforços de classificação, o conhecimento a respeito desse 
tipo de ferramenta ainda é restrito. Dessa forma, é necessário a elaboração de estratégias que possam popularizar ou ampliar o alcance dessas iniciativas perante a sociedade.

\par
Alinhado a essa necessidade, o objetivo desse trabalho é desenvolver uma aplicação \textit{web}, denominada e-TAPE, para apoiar a edição colaborativa de uma taxonomia sobre 
ferramentas de participação eletrônica. O nome atribuído à taxonomia, que é base para ferramenta, é \acrfull{tape}, nome dado pelos autores \citeonline{tape2019mota}. 
A proposta é que a e-TAPE facilite a interação dos usuários de maneira intuitiva e amigável e funcione como um canal para que, tanto pesquisadores quanto potenciais usuários de
ferramentas de participação eletrônica, colaborem para evolução da taxonomia. Na e-TAPE, os usuários têm acesso à taxonomia e às ferramentas de participação eletrônica já classificadas.
É esperado que a ferramenta contribua tanto para a pesquisa quanto para a prática do assunto abordado. 

\par
No que diz respeito à pesquisa, a e-TAPE pode ajudar os pesquisadores a entender mais sobre domínio estudado. E também, a utilização de um modelo sistemático representado pela taxonomia 
pode facilitar a identificação dos pontos fortes e fracos, criando novas possibilidades de investigação. Por outro lado, tanto para o gestor público, quanto para o cidadão comum, 
o uso da e-TAPE pode facilitar a escolha de ferramentas de participação eletrônica que se adequem mais ao seu contexto. Essa escolha poderá levar em conta critérios 
isolados ou uma combinação de critérios, dependendo da demanda corrente. 

\par
A usabilidade da e-TAPE é avaliada, seguindo metodologias já validadas no meio acadêmico e os resultados indicam que, a e-TAPE possui um bom indicador de usabilidade perante aos 
usuários que realizaram o teste. 

\par
A estruturação do texto deste projeto dá-se da seguinte forma: no Capítulo \ref{cap:cap2}, é apresentada a revisão da literatura junto dos principais conceitos para uma melhor 
compreensão do trabalho realizado. No capítulo \ref{cap:cap3}, retrata-se a metodologia de desenvolvimento da aplicação, as classes de usuários, funcionalidades disponíveis e
as tecnologias utilizadas para a implementação da ferramenta. Já no capítulo \ref{cap:cap4}, encontram-se a avaliação da usabilidade da ferramenta e os resultados obtidos. 
A conclusão e as sugestões para futuros trabalhos encontram-se no capítulo \ref{cap:cap5}.
