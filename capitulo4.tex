\chapter[Avaliação]{Avaliação}
\label{cap:cap4}

Neste capítulo são apresentados a avaliação da ferramenta, os resultados obtidos, as análises das informações levantadas e a metodologia utilizada para realizar a avaliação. 

% 1a seção do capítulo 4
\section{Cenário da Avaliação}
\label{sec:cenario}
O objetivo desta avaliação foi validar a ferramenta desenvolvida com a utilização por possíveis usuários finais. Essa avaliação deu-se via internet e foi dividida
em três etapas: obtenção das informações gerais dos usuários, realização de tarefas pré-determinadas e avaliação da usabilidade da ferramenta.

%Falar da ISO

\par
Na primeira etapa, os usuários participantes da avaliação foram informados quanto ao objetivo do experimento, orientados sobre os procedimentos a serem seguidos e questionados 
sobre informações pessoais, como por exemplo: idade, sexo, facilidade com o uso de tecnologias e sobre seu interesse por política e participação da sociedade civil organizada
nas decisões tomadas pelo poder público.

\par
Após a primeira etapa e para que a avaliação de usabilidade fosse minuciosa e não exaustiva, os usuários foram separados, de forma aleatória, em dois grupos.
Foi entregue a cada usuário do primeiro grupo um conjunto de dados contendo cinco ferramentas de participação eletrônica e suas descrições textuais.
A partir dessas descrições, cada usuário classificou a ferramenta em questão de acordo com o que achou mais adequado.
Essa abordagem foi adotada para que fosse possível a comparação entre a classificação de uma ferramenta de participação feita por um especialistas no assunto em comparação com
a mesma ferramenta de participação classificada por usuários leigos.
Já aos usuários do segundo grupo, foi entregue um conjunto de perguntas sobre as ferramentas previamente classificadas. Essa abordagem foi adotada para que fosse possível avaliar
o grau de assertividade na busca por um determinado tipo de ferramenta de participação eletrônica.

\par
Na terceira etapa, o objetivo foi avaliar a usabilidade a ferramenta desenvolvida sob os aspectos de efetividade, satisfação, eficiência de uso e aprendizagem. 
As questões de usabilidade foram elaboradas usando como base os trabalhos de "BARBARA" e "FERNANDA" e a ISO/IEC 25010