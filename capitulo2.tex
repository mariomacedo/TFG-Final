\chapter[Revisão Bibliográfica]{Revisão Bibliográfica} 
\label{cap:cap2}
%Eu acho que você quis associar o conceito de participação com governança e transparência, acho essa abordagem muito legal. 
%Mas acho que falta você trabalhar um pouco mais essa relação. Pensa que deve existir uma coerência nessa discussão:
%minha sugestão é que você desenvolva melhor o texto seguindo esse caminho: governança digital->transparência->participação

% Introdução do capítulo
Neste capítulo são apresentados os tópicos teóricos necessários para uma boa compreensão do desenvolvimento deste trabalho.

% 1a seção do capítulo 1
\section{Participação Eletrônica}
\label{sec:e-part}
Atualmente, fatores como a sociedade civil organizada sendo reconhecida pelos governos e governantes, e o consequente aumento da participação da população nos
programas governamentais, junto com a ampliação e diversificação dos temas abordados nas esferas governamentais, têm determinado um novo modelo de governança \cite{o2011government}. 
Este modelo acaba por oferecer um espaço, antes não existente,  onde o conceito de participação pode ser ampliado até o conceito de cidadania, ou seja, a participação cidadã.

\par
De acordo com a \acrshort{onu}, promover a participação cidadã é fundamental para a governança de uma sociedade civil organizada e inclusiva.
O objetivo dessa participação deve ser composto pela melhoraria do acesso às informações e aos serviços públicos,
e pelo incentivo a inclusão do cidadão nas tomadas de decisões públicas que impactem o bem estar da sociedade como um todo, e do indivíduo em particular. 
As funções de fazer política e entregar serviços públicos precisam ser reinterpretadas e os cidadãos têm que se envolver nos processos políticos \cite{bovaird2007beyond}.

\par
Sendo assim, a incorporação de novas tecnologias na relação entre estado e sociedade amplia e possibilita uma outra dinâmica a essa relação. 
Seguindo o conceito de governança digital, apresentado pela \acrshort{onu}, que trata do uso de \acrfull{tic} para atender três pilares fundamentais,
a prestação de melhores serviços pelos agentes públicos, o maior acesso a informação e a maior participação da sociedade civil em todos os ciclos de políticas públicas.
As \acrshort{tic} permitem a alavancagem desses três pilares em um alto patamar de desempenho. 

\par
A noção de governança digital é interpretada por \citeonline{reddick2012public} como o resultado da adoção de infraestruturas tecnológicas 
para a otimização da entrega de serviços à população. Ou seja, a utilização de \acrshort{tic} pelo setor público, passa a se referir ao modo como as tecnologias e a
internet podem melhorar a capacidade do Estado de formular e implementar suas políticas públicas \cite{parra2017governancca}.

\par
Deste modo, \citeonline{germani2016desafios} complementa a definição de governança digital afirmando que sua utilização pelo setor público,
deve ter o objetivo de melhorar a disponibilização de informações e o fornecimento de serviços públicos, além de incentivar o engajamento
da sociedade nos processos políticos, aprimorando os níveis de responsabilidade, transparência e efetividade do governo.

\par
A utilização de \acrshort{tic} pelo setor público tem transformado a governança global como um todo, exigindo que governos e governantes forneçam
serviços melhores, mais econômicos e eficientes às organizações e indivíduos \cite{afdb2014uneca}. 

\vspace{0.3CM}

\par
O conceito de participação eletrônica foi definido por \citeonline{macintosh2008democracy} como o uso de informações e comunicação tecnológicas a fim de ampliar e aprofundar
a participação política para que cidadãos sejam capazes de se conectar uns aos outros e aos seus representantes eleitos.
Para \citeonline{braga2016participaccao}, a possibilidade de um maior acesso as informações e ao conhecimento, permite uma maior transparência nas decisões tomadas por governantes.
\citeonline{vaz2017transformaccoes} argumenta que há a necessidade de alteração do atual modelo
\textit{broadcasting}, para um modelo mais participativo de tomada de decisões.

\par
Define-se "modelo \textit{broadcasting}" como um modelo de alocação dos recursos digitais como recursos secundários ou complementares às iniciativas presenciais
de relacionamento entre governos e sociedade. Nesse modo, são os governantes que estabelecem os momentos, formatos e conteúdo dos processos participativos e de controle social.
Isso acaba por restringir as iniciativas apenas às iniciativas governamentais, nas quais a interação e participação nas decisões e no controle social das políticas públicas 
são monopolizadas pelo Estado \cite{parra2017governancca}.

\par
Para que haja a evolução desse modelo atual, \citeonline{o2011government} sugere que o governo deve disponibilizar suas informações em uma infraestrutura que permita
a sua reutilização sistemática pela sociedade civil. Ao disponibilizar suas informação, o governo estimula o desenvolvimento de inciativas e ferramentas tecnológicas,
ampliando assim a possibilidade do uso diverso das informações \cite{zuiderwijk2012socio}.

\par
Segundo \citeonline{vaz2017transformaccoes}, o crescimento do conceito e da comunidade \textit{open source}, atrelados ao compartilhamento de dados governamentais,
permite a produção descentralizada de aplicações, serviços e sistemas tecnológicos. Criando assim, o que o autor chama de "segunda geração de governança digital", 
que consiste na busca por iniciativas além das unidirecionais(\textit{broadcasting}) , onde o governo apenas disponibiliza os dados captados sobre a população.

\par
O conceito de governança digital \textit{open source} democratiza a tomada de decisão e incentiva a colaboração voluntária entre os indivíduos \cite{rushkoff2003open}.
Diversos paradigmas, sobre este modelo de tomada de decisão, vêm sendo reexaminados, e o papel do gestor público e do cidadão têm sido reavaliados.
O uso das ferramentas de participação eletrônica tem se disseminado para diversos casos de uso e em muitos ambientes diferentes \cite{medeiros2009novos}.

%!!!!!!!!!!!!!!!!!!!!!!!!!!!!!!!!!!!!!!!!!

OS PROXIMOS 3 PARAGRAFOS FICAM NESSA SESSÃO?

%!!!!!!!!!!!!!!!!!!!!!!!!!!!!!!!!!!!!!!!!!

\par
São muito os exemplos de ferramentas de participação eletrônica, contudo, a sociedade civil ainda não têm se engajado plenamente ao uso de todos os modelos ferramentas.
Para exemplificar, temas de grande interesse popular como as reformas trabalhista, tributária e da previdência social brasileira, recebem grande atenção dos 
veículos de comunicação, são diversas vezes citadas propostas de governantes pleiteando cargos e são relatadas por especialistas como fundamentais para o desenvolvimento do país.
Seguindo o raciocínio, espera-se um engajamento relativamente grande perante esses assuntos, porém, a ferramenta de participação eletrônica Wikilegis, desenvolvida pela câmara
dos deputados do Brasil, que disponibiliza um ambiente onde a sociedade civil pode analisar os projetos de leis e contribuir com sugestões
de nova redação a artigos ou parágrafos, que serão acompanhadas pelos relatores das proposições, conta com números pífios de sugestões para os projetos de reformas.
São 221 sugestões para a reforma da previdência, 129 sugestões para a reforma tributária e míseras 50 sugestões para a reforma trabalhista. 
Os números mostram a dificuldade de engajar a sociedade civil em assuntos políticos mais complexos e que demandam tempo e estudo.

\par
Por outro lado, o site Votanaweb conta com um painel onde são mostrados projetos de leis, seus propositores e componentes gráficos para representar a votação dos cidadãos.
Assim, o cidadão pode comparar sua intenção de voto com as dos demais cidadãos e com os votos efetivados pelos representantes, além de contar com a descrição dos votos por gênero, 
região e idade. Os projetos têm em média sete mil votos, onde o mais votado conta com certa de quase quinze mil votos. 

\par
Isso indica que o engajamento da sociedade civil organizada tem um amplo espaço de crescimento \cite{o2011government}, fica o desafio aos atores, sociedade civil, governantes, 
empresas e indivíduos, à criação de ferramentas de participação eletrônica que estimulem o engajamento do cidadão.

\newpage
\section{Ferramentas de Participação Eletrônica}
\label{sec:e-part tools}
Devido aos incentivos gerados por esses novos paradigmas de tomada de decisões participativas, muitas ferramentas de participação eletrônica têm sido desenvolvidas
tanto por governos e governantes, quanto pela sociedade e o mercado. 

\par
Alguns exemplos podem ser encontrado na Tabela \ref{tab:ferramentas}.

\begin{table}[!ht]
    \centering
    \caption{Ferramentas de Participação Eletrônica}
    \label{tab:ferramentas}
    \begin{tabular}{l*{2}{>{\raggedright\arraybackslash}p{0.5\linewidth}}}
    \toprule
        Nome                             & Acesso                           \\ 
    \midrule
        Crowd For Roads                  & c4rs.eu                          \\
        Decidim Barcelona                & decidim.barcelona                \\
        e-Cidadania                      & senado.leg.br/ecidadania         \\
        Iniciativa de Cidadania Europeia & ec.europa.eu/citizens-initiative \\
        LabRIO                           & lab.rio                          \\
        Mandato Participativo            & saopaulo.sp.leg.br               \\
        Participa.BR                     & participa.br                     \\
        Patio                            & patiolla.fi                      \\
        Plataforma Brasil                & plataformabrasil.org.br          \\
        Portal e-Democracia              & edemocracia.camara.gov.br        \\
        SigaLei                          & sigalei.com.br                   \\
        Visor Urbano                     & visorurbano.com                  \\
        Vote na Web                      & votenaweb.com.br                 \\ 
        We The People                    & petitions.whitehouse.gov         \\
        WeLive                           & welive.eu                        \\
    \bottomrule
    \end{tabular}
\end{table}

\newpage

As novas tecnologias mostraram uma gama de possibilidades para que os cidadãos ampliem o peso de sua participação nas decisões políticas,
melhorando a capacidade de mobilização, articulação, e possibilitando um maior envolvimento dos atores sociais \cite{araujo2015democracia}.\\

\par
\citeonline{saebo2008shape} dividem os atores sociais em quatro grupos: 

\begin{minipage}{.66\textwidth}	
   \textit{I}) Cidadãos, \\
   \textit{II}) Governantes, \\
   \textit{III}) Instituições estatais e \\
   \textit{IV}) Instituições Voluntárias. \\
\end{minipage}

\par
Esse agrupamento feito pelo autor supracitado é uma abstração da análise de \citeonline{macintosh2006evaluating}, deixando de lado um quinto grupo:

\par
\textit{V}) Provedores de Tecnologia.

\par
Contudo, \citeonline{wimmer2007ontology} agrupa esses atores mais objetivamente, dividindo-os em apenas dois grupos, são eles:\\

\begin{minipage}{.75\textwidth}	
   \textit{a}) Beneficitários da utilização das ferramentas e \\
   \textit{b}) Responsáveis pela administração da ferramenta de participação.  \\
\end{minipage}

\par
Demonstrando uma necessidade e interesse em ferramentas de participação eletrônica, a \acrfull{ocde}, juntamente com a \acrshort{onu} e a \acrfull{ue} criaram o \acrfull{epi}, 
índice que avalia governos frente ao uso de serviços digitais para prover informação aos seus cidadãos, interagir com os atores da sociedade e engajar os cidadãos no processo 
de tomada de decisão. 

\par 
O \acrshort{epi} de um país é composto pelos mecanismos de participação eletrônica que são utilizados pelo governo, comparado relativamente a todos os outros países.
O intenção desse índice não é de indicar práticas a serem seguidas, mas sim oferecer um parâmetro geral para que diferentes países possam saber o quanto os governos têm
feito uso e promovido iniciativas de participação eletrônica.

\par
Diante deste cenário, onde o uso de ferramentas de participação eletrônica está sendo incentivado por instituições, governos e cidadãos, a criação de uma taxonomia para 
classificação de ferramentas de participação eletrônica de maneira colaborativa, demonstrou-se interessante e necessária.*


%==================================================================================================================================================================================================

\section{Taxonomia}
\label{sec:taxonomia}
Com o objetivo de apresentar um panorama conceitual sobre o termo taxonomia e contextualizar o assunto, \citeonline{novo2007elaboraccao} diz que o conceito de taxonomia
não é algo moderno e não emerge instantaneamente como um modelo de solução de problemas de representação do conhecimento sobre um dado domínio. Mas é sim,
o resultado de um extenso processo histórico de estudos e investigações que convergiram para uma construção teórica sobre o assunto. 

\par
O termo taxonomia se origina do grego \textit{taxis} (ordem) e \textit{nomos} (lei, norma) e foi utilizado pela primeira vez em 1735 pelo cientista e médico sueco Carl von Linné,
com a publicação da obra \textit{Systema Naturae}, que contava com apenas 10 páginas em sua primeira edição. Em 1770, em sua 13ª edição, a obra já contava com mais de 3000 páginas.
Durante o século XVIII, Linné classificou os seres vivos de acordo com suas características distintivas e os hierarquizou, dividindo-os em reinos,
filos, classes, ordens, famílias, gêneros e espécies, que após algum tempo foram subdivididos. Sua classificação ficou conhecida como “Taxonomia de Lineu”.

\par
Tradicionalmente utilizada para a classificação das espécies em botânica e zoologia, taxonomias têm sido também utilizadas para estudos da área de Ciência da Informação.
\citeonline{terra2005taxonomia} definem o uso de taxonomias nesse contexto como um instrumento ou elemento de estrutura que permite alocar,
recuperar e comunicar informações dentro de um sistema de maneira lógica.

\par
A utilização de taxonomia nos sistemas de informação não leva em consideração família, gênero ou espécie, mas sim conceitos.
As classes e subclasses de uma taxonomia se apresentam de maneira lógica, suportada por princípios classificatórios.\cite{campos2012taxonomia}
\citeonline{aganette2010elementos} diz que diferente do princípio dicotômico adotado na taxonomia dos seres vivos, atualmente, faz-se possível a construção de taxonomias
policotômicas, ou seja, onde um objeto é classificado em tantas classes, e subclasses quantas necessárias, dentro um domínio especializado.

\par
Sendo assim, é possível utilizar a representação taxonômica da Figura \ref{fig:exemploTaxonomia} para classificar um objeto, arbitrariamente escolhido,
de acordo com os nós F e C, por exemplo.

\begin{figure}[!ht]
\begin{tikzpicture}[level 1/.style={sibling distance=5cm},level 2/.style={sibling distance=2.5cm}]
    \node {T}
        child { node {A}
        child {node{D}}
        child {node{E}}
        child {node{F}}
        child {node{G}}
        child {node{H}}
        }
        child { node {B}}   
        child { node {C}}  
    ;
\end{tikzpicture}
\caption{Exemplo de Taxonomia}
\label{fig:exemploTaxonomia}  
\end{figure}

%==================================================================================================================================================================================================
\subsection{Taxonomia para Ferramentas de Participação Eletrônica}
\label{subsec:taxonomia e-part tools}
\par
Encontramos durante a fase de revisão bibliográfica alguns estudos e ações no sentido de tentar elaborar tipos de classificações para as ferramentas de participação.
Como por exemplo, o projeto \textit{Civic Tech Field Guide}, que ainda se encontra em estado embrionário, sendo o principal meio de colaboração o preenchimento de um formulário online.

\par
Na seção \ref{subsec:taxonomiaElaborada}, será explicitado o modelo taxonômico adotado por este trabalho.

%==================================================================================================================================================================================================
%\subsection{Taxonomia para Outras Ferramentas}
%\label{subsec:taxonomia other tools}

%==================================================================================================================================================================================================
\subsection{Taxonomia para Outras Áreas da Computação}
\label{subsec:taxonomia other computer areas}
Taxonomia, no contexto da computação, tem aplicação em distintas áreas. A utilização de estruturas taxonômicas para classificar sistemas, arquiteturas e arquivos data de mais de
cinquenta anos. Uma das classificações taxonômica com maior relevância para a área da computação é a chamada "Taxonomia de Flynn",
onde \citeonline{flynn1966very} classificou as arquiteturas de computadores da seguinte forma:\\

\begin{minipage}{.66\textwidth}
    \begin{singlespace}
        \begin{itemize}
            \item \acrfull{sisd}
            \item \acrfull{simd}
            \item \acrfull{misd}
            \item \acrfull{mimd}
        \end{itemize}
    \end{singlespace}
\end{minipage}
\vspace{0.5cm}

\begin{figure}[!ht]
    \begin{tikzpicture}[level 1/.style={sibling distance=5cm},level 2/.style={sibling distance=2.5cm}]
        \node {T}
            child { node {\acrshort{sisd}}}   
            child { node {\acrshort{simd}}}  
            child { node {\acrshort{misd}}}   
            child { node {\acrshort{mimd}}}  
        ;
    \end{tikzpicture}
    \caption{Taxonomia de Flynn}
    \label{fig:taxonomiaFlynn}  
\end{figure}

\vspace{0.5cm}
\par
A área de tolerância a faltas, da engenharia de \textit{software}, tem taxonomias comumente adotadas para definir termos e técnicas.
Entre as mais conhecidas estão a taxonomia proposta por \citeonline{gartner1999fundamentals}, abordando diversos conceitos e os aplicando a um cenário distribuído,
e a taxonomia apresentada por \citeonline{avizienis2004basic}, definindo conceitos sobre tolerância a faltas e segurança computacional.

\par
\citeonline{sondhi2018taxonomy} criou uma taxonomia que pode ser usada para prever as ações ou intenções de um usuário em particular de uma dada loja virtual
e então personalizar o algoritmo de busca para indicar as necessidades específicas desse usuário.

\par
Vale ressaltar a grande utilização de taxonomia por lojas virtuais, e o grande número de trabalhos encontrados sobre taxonomia aplicada a esse setor.

%==================================================================================================================================================================================================
\subsection{Taxonomia Elaborada}
\label{subsec:taxonomiaElaborada}
\par
Por meio do projeto Vispública TODO: CITAR O ARTIGO, foi elaborada a taxonomia utilizada neste trabalho. O realizado pelo projeto Vispública foi no sentido de fornecer
a classificação de ferramentas de participação através de uma taxonomia, de forma que seja possível associar uma ferramenta às dimensões propostas e também permitir
uma compreensão geral do que existe acerca das iniciativas de participação eletrônica identificadas na literatura.

\par
Essa Taxonomia, mostrada na Figura \ref{fig:taxonomia-vispublica}, é composta por 4 grupos: sustentação, domínio, tecnologias e funcionalidades.
Os grupos são diferenciados por cores, e para cada grupo, identificaram classes e/ou subclasses.

\begin{figure}[!ht]
    \centering
    \includegraphics[scale=0.30]{./figuras/taxonopart-radial.png}
    \caption{Taxonomia elaborada pelo projeto Vispública}
    \label{fig:taxonomia-vispublica}
\end{figure}


\subsection{Grupos, classes e subclasses}
\label{subsec:gruposClassesSubclasses}

\par
Nesta seção serão mostradas as definições de cada grupo, classe e subclasse, de acordo com TODO: Citar o artigo


%========================================================================================================================================================================================
\subsubsection{Sustentação}
\label{subsubsec:sustentacao}
O grupo sustentação explora as questões relacionadas a como e quem promove a ferramenta de forma que seja utilizada, ou seja, como se dá a sustentabilidade da ferramenta.
Na Figura \ref{fig:grupo-sustentacao} é possível observar a hierarquia para classificação definida no grupo sustentação.

\begin{figure}[!ht]
    \centering
    \includegraphics[scale=0.20]{./figuras/sustentacao.png}
    \caption{Hierarquia do grupo sustentação}
    \label{fig:grupo-sustentacao}
\end{figure}

\par
As definições das classe e subclasse desse grupo são representadas e descritas na Tabela \ref{tab:classesSustentacao}.

\begin{table}[!ht]
    \centering
    \caption{Classes e Subclasses do Grupo Sustentação}
    \label{tab:classesSustentacao}
    \begin{tabular}{l*{2}{>{\raggedright\arraybackslash}p{0.5\linewidth}}}
    \toprule
        Nome         & Descrição                       \\ 
    \midrule
        Abordagem    & Indica a origem da iniciativa.\\                         
        Autoria      & Informa quem é o responsável pelo desenvolvimento ou pela iniciativa da ferramenta.                \\
        Engajamento  & Indica se foi utilizada algum método específico na tentativa de aumentar o engajamento.         \\
        Técnica      & Refere-se à utilização de alguma técnica já estruturada para aumentar o engajamento dos usuários. \\
        Estratégia   & Diz respeito as ações realizadas na tentativa de aumentar o engajamento.\\
    \bottomrule
    \end{tabular}
\end{table}


%========================================================================================================================================================================================
\newpage
\subsubsection{Domínio}
\label{subsubsec:dominio}
O grupo domínio tem o objetivo de considerar características do ambiente em que a ferramenta está inserida. 
A figura \ref{fig:grupo-dominio} representa de forma visual as hierarquias definidas para o grupo.

\begin{figure}[!ht]
    \centering
    \includegraphics[scale=0.20]{./figuras/dominio.png}
    \caption{Hierarquia do grupo domínio}
    \label{fig:grupo-dominio}
\end{figure}

\par
As definições das classe e subclasse desse grupo são representadas e descritas na Tabela \ref{tab:classesDominio}.

\begin{table}[!ht]
    \centering
    \caption{Classes e Subclasses do Grupo Domínio}
    \label{tab:classesDominio}
    \begin{tabular}{l*{2}{>{\raggedright\arraybackslash}p{0.5\linewidth}}}
    \toprule
        Nome                  & Descrição \\ 
    \midrule
        Área                  & Indica se a ferramenta está inserida em alguma área específica da participação.\\                         
        Localização           & Informa se a ferramenta está inserida em um local geográfico específico e identifica qual o local.                \\
        Esfera governamental  & Identifica em qual nível da esfera governamental a ferramenta está inserida.       \\
        Idioma                & Informa em quais idiomas a ferramenta é disponibilizada.\\
        Público alvo          & Caracteriza qual público alvo que a ferramenta busca atingir e engajar.\\
        Tipo de participação  & Define se a participação é voluntária ou involuntária \\
    \bottomrule
    \end{tabular}
\end{table}

%========================================================================================================================================================================================
\newpage
\subsubsection{Tecnologias}
\label{subsubsec:tecnologias}
O grupo tecnologia contempla a análise sobre os aspectos técnicos utilizados nas ferramentas.
A figura \ref{fig:grupo-tecnologias} representa de forma visual as hierarquias definidas para o grupo.

\begin{figure}[!ht]
    \centering
    \includegraphics[scale=0.20]{./figuras/tecnologias.png}
    \caption{Hierarquia do grupo tecnologias}
    \label{fig:grupo-tecnologias}
\end{figure}

\par
As definições das classe e subclasse desse grupo são representadas e descritas na Tabela \ref{tab:classesTecnologias}.

\begin{table}[!ht]
    \centering
    \caption{Classes e Subclasses do Grupo Tecnologias}
    \label{tab:classesTecnologias}
    \begin{tabular}{l*{2}{>{\raggedright\arraybackslash}p{0.5\linewidth}}}
    \toprule
        Nome                      & Descrição \\ 
    \midrule
        Plataforma                & Indica em qual plataforma a ferramenta é disponibilizada.\\
        Hardware                  & Informa se a ferramenta utiliza algum hardware em sua implementação e qual hardware é esse.\\
        Desenvolvimento           & Descreve os itens associados aos recursos utilizados para o desenvolvimento da ferramenta. \\
        Banco de dados            & Refere-se ao uso de algum sistema de armazenamento de dados.\\
        Servidor web              & Indica se a ferramenta utiliza servidores e qual o tipo de servidor utilizado.\\
        Linguagens de Programação & Identifica quais linguagens foram utilizadas no desenvolvimento da ferramenta. \\
        Bibliotecas               & Indica se a ferramenta faz uso de alguma biblioteca de código.\\
        API                       & Informa se a ferramenta faz integração com alguma API. \\
    \bottomrule
    \end{tabular}
\end{table}

%========================================================================================================================================================================================
\newpage
\subsubsection{Funcionalidades}
\label{subsubsec:funcionalidades}
O grupo funcionalidades aborda as principais características de utilização disponibilizadas para interação com o usuário.
A hierarquia proposta para o grupo Tecnologias está representada na figura \ref{fig:grupo-funcionalidades}.

\begin{figure}[!ht]
    \centering
    \includegraphics[scale=0.30]{./figuras/funcionalidades.png}
    \caption{Hierarquia do grupo funcionalidades}
    \label{fig:grupo-funcionalidades}
\end{figure}

\par
As definições das classe e subclasse desse grupo são representadas e descritas na Tabela \ref{tab:classesFuncionalidades}.

\begin{table}[!ht]
    \centering
    \caption{Classes e Subclasses do Grupo Funcionalidades}
    \label{tab:classesFuncionalidades}
    \begin{tabular}{l*{2}{>{\raggedright\arraybackslash}p{0.5\linewidth}}}
    \toprule
        Nome                       & Descrição \\ 
    \midrule
        Visualização de Informação & Indica se é utilizada alguma técnica de visualização para a apresentação das informações e quais informações são apresentadas pela ferramenta.\\
        Técnica                    & Refere-se à qual técnica de visualização é utilizada para representações gráficas.\\
        Informação                 & Diz respeito das informações que são disponibilizadas pelas visualizações.\\
        Coleta de Dados            & Define informações sobre os dados coletados e como são utilizados.\\
        Tipo de Dado               & Define os tipos de dados que são coletados.\\
        Estratégia                 & Relaciona-se a forma como os dados são coletados.\\
        Processamento de Dados     & Identifica se existe alguma forma de processamento sobre os dados.\\
        Dados abertos              & Indica se há a utilização de dados abertos, duas possibilidades podem ocorrer, os dados gerados pela ferramenta são disponibilizados em formato aberto, ou então, esses dados são utilizados para complementar alguma informação gerada pela ferramenta.\\
        Tipo de Informação         & Determina a respeito de que as informações na ferramenta tratam-se.\\
        Interação entre usuários   & Demonstra, para as aplicações que tem interação entre os usuários, qual o objetivo desta interação e qual a técnica utilizada.\\
        Objetivo                   & Refere-se ao objetivo da interação entre os usuários na aplicação.\\
        Técnica	                   & Define a forma como se dá a interação entre os usuários.\\
        Moderação                  & Apresenta quem é responsável por moderar as informações na ferramenta.\\
        Direcionamento             & Identifica quais ferramentas realizam encaminhamento de informações ao poder público.\\
        Autenticação               & Informa se existe a exigência de autenticação do usuário para utilizar a ferramenta.\\
    \bottomrule
    \end{tabular}
\end{table}
