\chapter[Conclusão e Trabalhos Futuros]{Conclusão e Trabalhos Futuros}
\label{cap:cap5}

O objetivo deste trabalho foi o desenvolvimento de uma aplicação \textit{web} com o intuito de servir como meio para a edição colaborativa de uma taxonomia
sobre ferramentas de participação eletrônica. Essa taxonomia em questão foi apresentada por \citeonline{tape2019mota}.

\par
Como já abordado no começo da seção \ref{sec:desenvolvimento}, a utilização do modelo interativo de desenvolvimento permite que requisitos sejam propostos e avaliados para serem
implementados em futuras versões. A versão atualmente disponível da e-TAPE foi avaliada, através de um questionário SUS, e obteve um \textit{Score SUS}, referente a análise de usabilidade,
de 78,75 em uma escala entre 0 e 100. O autor \citeonline{sauro2015supr} estabelece uma média de 68 pontos como referência para a avaliação de ferramentas. 
Assim, constata-se que para uma primeira versão, e-TAPE conseguiu agradar ao grupo que a avaliou. 

\par
A avaliação também gerou dados sobre os componentes de avaliação da qualidade da ferramenta, onde a e-TAPE ficou acima da média em 3 dos 5 critérios de avaliação. Com isso foram
identificadas algumas limitações, mas o resultado de maneira geral foi satisfatório. 

\par
Algumas funcionalidades já estão previstas para serem implementadas, como por exemplo, cadastro de usuários; criação de uma classe de usuários moderadores,
para que assim as ferramentas possam ser editadas de uma maneira moderada. Outra funcionalidades que deverão ser implementadas é a integração com redes sociais, para compartilhamento, 
a adição de comentários e sugestão de evolução da taxonomia.

\par
Espera-se que este projeto e a aplicação e-TAPE contribua com pesquisadores, gestores públicos e sociedade civil. Para que as iniciativas de ferramentas de participação eletrônica
sejam amplamente difundidas e usadas. 

\par
Como trabalhos futuros, almeja-se finalizar a implementação dos requisitos levantados durante a elaboração deste trabalho, para que então, seja possível a redação de artigos científicos
para a disseminação do conhecimento obtido. 