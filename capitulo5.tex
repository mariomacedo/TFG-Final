\chapter[Conclusão e Trabalhos Futuros]{Conclusão e Trabalhos Futuros}
\label{cap:cap5}

Neste trabalho, foi desenvolvida uma aplicação \textit{web} com o intuito de servir como meio para a edição colaborativa de uma taxonomia
sobre ferramentas de participação eletrônica. Essa taxonomia em questão foi elaborada e denominada como TAPE por \citeonline{tape2019mota}. 
O desenvolvimento da ferramenta e-TAPE é um projeto que se encontra em andamento, e o objetivo deste trabalho foi a construção de uma primeira versão útil dessa aplicação, 
de modo que o projeto pudesse continuar sua evolução.

\par
Como já abordado na seção \ref{sec:desenvolvimento}, a utilização do modelo interativo de desenvolvimento permite que requisitos sejam propostos e avaliados para serem
implementados em futuras versões. A versão atualmente disponível da e-TAPE foi avaliada, através de um questionário SUS, e obteve um \textit{Score SUS}, referente a análise de usabilidade,
de 78,75 em uma escala entre 0 e 100. O autor \citeonline{sauro2015supr} estabelece uma média de 68 pontos como referência para a avaliação de ferramentas. 
Assim, constata-se que para uma primeira versão, e-TAPE conseguiu atingir uma boa avaliação. 

\par
A avaliação também gerou dados sobre os componentes de avaliação da qualidade da ferramenta, onde a e-TAPE ficou acima da média em 3 dos 5 critérios de avaliação. Com isso foram
identificadas algumas limitações, mas o resultado, de maneira geral, foi satisfatório. 

\par
Além das funcionalidades associadas à facilidade de uso e satisfações identificadas na avaliação, outras funcionalidades já estão previstas para serem implementadas, como: 
permitir o cadastro e o controle de acesso por diferente classes de usuários (moderadores, cidadãos e visitantes), permitir a edição da taxonomia de forma moderada, 
integração com redes sociais para compartilhamento e adição de comentários e sugestão de evolução da taxonomia. 


\par
Espera-se que este projeto e a aplicação e-TAPE contribua com pesquisadores, gestores públicos e sociedade civil para que as iniciativas de ferramentas de participação eletrônica
sejam amplamente difundidas e utilizadas. 

