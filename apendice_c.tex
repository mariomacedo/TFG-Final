\chapter{Descrição das ferramentas entregue aos participantes do estudo.}
\label{apendice:c}

\par
Neste apêndice encontram-se os textos submetidos como tarefa aos participantes do estudo.

\section{Texto 1}
\label{sec:text1}
Colab.re

Colab.re é uma ferramenta de participação eletrônica construída como uma rede social na tentativa de aumentar a quantidade de usuários, consequentemente a participação e o engajamento. 
Na ferramenta,  os usuários podem se cadastrar e se inscrever em determinadas cidades para visualizar e inserir problemas. Não podem ser inseridos problemas das esferas estaduais, 
regionais e federais, somente municipais. Na inserção do problema, é possível adicionar vários tipos de dados, como foto, a localização e uma descrição textual. Isso é feito através
de um formulário disponibilizado na própria página da ferramenta. Os usuários visualizam os problemas inseridos na cidade e podem se comunicar através de mensagens de texto com outros
usuários para discutir os problemas e propor soluções. A ferramenta também  disponibiliza módulos específicos para os gestores que podem responder usuários, criar enquetes, 
visualizar os problemas cadastrados em mapas e outros gráficos. A ferramenta é divulgada no Facebook e em um blog no qual são disponibilizados conteúdos sobre participação
cidadã e novidades da ferramenta. Essa estratégia é usada para aumentar o engajamento da ferramenta. A rede social está disponível online  em www.colab.re ou por
aplicativos móveis. A ferramenta foi construída, tanto para plataforma web quanto para mobile, utilizando as seguintes tecnologias: o banco de dados MongoDB, a linguagem de 
programação Javascript junto das bibliotecas Jquery, Angular, MomentJs, Highcharts e Modernizr. Além de contar com consumo de serviços de APIs do Google Maps e Facebook. 
Sabe-se que a ferramenta é hospedada em servidores que utiliza tecnologias como Node.js e Nginx. Os recursos da ferramenta são disponibilizados tanto em português quanto em inglês. 

\section{Texto 2}
\label{sec:text2}

Triang

Triang é uma ferramenta de participação eletrônica iniciada pela comunidade e orientada ao governo, a ferramenta foi criada pela Triang Inc. Com o intuito de aumentar engajamento dos
usuários, usou-se a técnica de gamificação. Para pontuar no jogo, o cidadão deve cadastrar problemas encontrados pela cidade, onde outros usuários podem interagir com ele, propondo
soluções aos problemas encontrados. Para cadastrar um problema, o usuário deve fazer login utilizando alguma rede social. Os jogadores são ranqueados, com a intenção de criar-se uma 
competição entre os jogadores, e assim, aumentar seu engajamento na ferramenta. Na inserção do problema, é possível adicionar vários tipos de dados, como foto, vídeo, a localização e 
uma descrição. Isso é feito através de um dispositivo móvel. Os usuários são notificados de problemas adicionados a 30km de sua localização. Há uma moderação dos problemas inseridos. 
Caso um problema receba avaliação negativa dos outros usuários, é alertado ao moderador para uma revisão do conteúdo. 
A ferramenta também disponibiliza módulos especiais para os gestores  públicos que podem responder aos usuários, criar enquetes, visualizar os problemas cadastrados em mapas e outros 
gráficos. Há a disponibilização das informações geradas abertamente à comunidade. 
A ferramenta é divulgada no Facebook, twitter e instagram, além de um canal no Youtube, onde é possível encontrar tutoriais para aprender a interagir com a ferramenta..
A ferramenta foi construída, somente para plataforma web, utilizando as seguintes tecnologias: o banco de dados PostgreSQL, a linguagem de programação Java junto das bibliotecas
Primefaces, Jquery, D3js e materialize. Além de contar com consumo de serviços de APIs do Google+, linkedin e Facebook. Sabe-se que a ferramenta é hospedada em servidores Wildfly. 
Os recursos da ferramenta são disponibilizados tanto em português, inglês, espanhol e alemão.
